\documentclass{beamer}

\usetheme{Warsaw}

\title[SORT]{Simple Online and Real-time Tracking for the Multi-object tracking Problem}
\author{Mazen Elaraby 19P9804 \\Muhammad Ahmed 19P6131}
\institute{Ain Shams University}
\date{\today}

\begin{document}

\begin{frame}
\titlepage
\end{frame}

\begin{frame}[t]{Multi-Object Tracking}
\begin{block}{Definition of Multi-Object Tracking}
Multiple object tracking (MOT) is a fundamental problem in computer vision that involves estimating the trajectories of multiple objects over time, given a sequence of observations from one or
more sensors.
\end{block}
\vspace{0.2 in}
\begin{columns}[onlytextwidth]
\column{0.5\textwidth} 
\includegraphics[width = 5 cm]{download.jpg}
\column{0.5\textwidth} 
\includegraphics[width = 5 cm]{images.jpg}
\end{columns}
\end{frame}

\begin{frame}[t]{Multi-Object Tracking}
\begin{itemize}
\item Traditional multiple object tracking algorithms like Multiple Hypothesis Testing (MHT) and
Joint Probabilistic Data Association (JPDA) suffer from high combinatorial complexity that can
lead to a combinatorial explosion.
\item The problem arises when the number of objects to be tracked
increases, as the algorithms need to consider all possible associations between detections and tracks,
leading to an exponential increase in computational requirements.
\end{itemize}
\vspace{0.2 in}
\begin{columns}[onlytextwidth]
\column{0.5\textwidth} 
\includegraphics[width = 5 cm]{Screenshot 2023-05-18 085037.png}
\column{0.5\textwidth} 
\includegraphics[width = 5 cm]{Implementation-process-of-joint-probability-data-association.png}
\end{columns}

\end{frame}

\begin{frame}[t]{Introduction to SORT}
\begin{itemize}
\item The Simple Online Real-time Tracking algorithm, commonly known as SORT, is a popular object
tracking algorithm that has gained significant attention in recent years due to its simplicity and
reliability
\item  SORT abides by the principle of Occam’s Razor,
which means it focuses on efficiency and simplicity rather than robustness against edge cases. 
\end{itemize}
\begin{center}
\includegraphics[width = 9 cm]{Overview-of-the-object-tracking-SORT-algorithm.png}
\end{center}
\end{frame}

\begin{frame}[t]{Detection}
\begin{itemize}
\item The detection phase is critical in the Simple Online Real-time Tracking (SORT) algorithm as it
provides the initial set of detections for tracking objects over time. 
\item To achieve accurate results, reliable methods must be used for detecting objects in a scene. Noisy or inaccurate detections can cause tracking algorithms to fail and result in incorrect object associations.
\end{itemize}
\begin{center}
\includegraphics[width = 7 cm]{f54e1b5b-de98-4e12-9f6f-4939194047e8_Resources_Real-time-Object.jpg}
\end{center}
\end{frame}

\begin{frame}[t]{Detection: YOLOv4}
\begin{itemize}
\item YOLOv4, a state-of-the-art object detection algorithm based
on deep neural networks, has shown exceptional performance in various benchmark evaluations
due to its accuracy and speed.
\item YOLO performs object detection, localization, and classification all at once
\item It uses a single-pass approach, making it faster and more efficient than older algorithms
\end{itemize}
\begin{center}
\includegraphics[width = 7 cm]{maxresdefault.jpg}
\end{center}
\end{frame}

\begin{frame}[t]{Detection: YOLOv4}
\begin{itemize}
\item The pipeline consists of feature extraction, learning, and non-maximal suppression stages
	\begin{itemize}
	\item Feature extraction involves using a CNN to extract relevant features from the input image
\item The learning stage predicts bounding boxes and class probabilities for each object in the image based on extracted features
\item Non-maximal suppression removes redundant bounding boxes and classifies the remaining detections.
	\end{itemize}
\end{itemize}
\begin{center}
\includegraphics[width = 6.5 cm]{Screenshot 2023-05-18 031313.png}
\end{center}
\end{frame}

\begin{frame}[t]{Estimation Model: Recursive Bayesian State Estimation
}
\begin{block}{Definition of the Bayes Filter}
Recursive Bayesian state estimation is a method for estimating the state of a dynamic system over time using probabilistic modeling. It is performed recursively over time using incoming measurements and a mathematical process model. 
\end{block}
\begin{itemize}
\item Recursive Bayesian state estimation is advantageous because it can handle uncertainty and noise in the system by representing the system's state as a PDF. 
\item This makes it useful in situations with random disturbances or measurement errors.
\end{itemize}

\end{frame}

\begin{frame}[t]{Estimation Model: Recursive Bayesian State Estimation}
The Bayes Filter uses a Prediction-Correction Mechanism.
\begin{itemize}
\item Prediction Step:
$$\overline{bel}(x_t) = \int p(x_t | u_t, x_{t-1}) .bel(x_{t-1}) dx_{t-1}$$ 
\item Correction Step:
$$bel(x_t) = \eta . p(z_t | x_t) .\overline{bel}(x_t)$$
\end{itemize}
\begin{center}
\includegraphics[width = 6.5 cm]{S2E3_Thumbnail.jpg}
\end{center}
\end{frame}

\begin{frame}[t]{Estimation Model: The Kalman Filter}
\begin{block}{Definition}
The Kalman Filter is a realization of the Bayes Filter for linear motion and observation models, and uses a Gaussian distribution to represent probability distributions. The filter provides an optimal estimate of the system state along with an estimate of uncertainty.
\end{block}

Estimated distributions by the Kalman filter assume the following form:  
$$p(x) = det(2\pi \Sigma)^{-1/2} \exp \left( -1/2 (x- \mu)^T \Sigma^{-1/2} (x- \mu) \right)$$
\begin{center}
\includegraphics[width = 3.5 cm]{2d_gaussian_distribution.png}
\end{center}
\end{frame}

\begin{frame}[t]{Estimation Model: The Kalman Filter}
Both the observation and motion model can be modeled as the following linear models:
	$$x_t = A_t x_{t-1} + B_t u_t + \epsilon_t$$
	$$z_t = C_t x_t + \delta_t$$
	
	where:
	$$A_t: \textrm{The State Transition Matrix}$$
	$$B_t: \textrm{The Control-input Model}$$
	$$C_t: \textrm{The Observation model}$$
	$$\epsilon_t \& \delta_t: \textrm{Random variables representing the process}$$
	$$ \textrm{and measurment noise with covariance } R_t \& Q_t$$
\end{frame}

\begin{frame}[t]{Estimation Model: The Kalman Filter}
	Motion under Gaussian noise leads to:
	$$p(x_t | u_t, x_{t-1}) = det(2\pi R_t)^{-1/2}$$
	$$ \exp \left( -1/2 (x_t - A_t x_{t-1} - B_t u_t)^T R^{-1} (x_t - A_t x_{t-1} - B_t u_t) \right)$$
	
	Measuring under Gaussian noise leads to:
	$$p(z_t | x_t) = det(2\pi Q_t)^{-1/2} \exp \left( -1/2 (z_t - C_t x_t)^T Q^{-1} (z_t - C_t x_t) \right)$$
\end{frame}

\begin{frame}[t]{Estimation Model: The Kalman Filter}
Plugging these distribution into the recursive Bayes filter yields the final form of the Kalman filter algorithm represented in the following pseudo-code:

$$\textrm{Kalman Filter}(\mu_{t-1}, \Sigma{t-1}, u_t, z_t)$$
\begin{align*}
\bar{\mu_t} &= A_t \mu_{t-1} + B_t u_t\\
\bar{\Sigma_t} &= A_t \Sigma{t-1} A_t^T + R_t\\
K_t &= \bar{\Sigma_t} C_t^T \left( C_t\bar{\Sigma_t} C_t^T + Q_t \right)^{-1}\\
\mu_t &= \bar{\mu_t} + K_t (z_t - C_t\bar{\mu_t})\\
\Sigma_t &= \left( I - K_tC_t \right)\bar{\Sigma_t}\\
\end{align*}	
$$\textrm{return } \mu_t, \Sigma_t$$
\end{frame}

\begin{frame}[t]{Data Association}
\begin{block}{Data Association}
Data association is the process of associating observations or measurements with existing tracks in a tracking algorithm.
\end{block}
\begin{itemize}
\item The process of assigning detections to existing targets involves estimating the location of each target's bounding box and calculating the IOU distance between each detection and all predicted bounding boxes from existing targets. This is used to determine the assignment cost matrix.
\end{itemize}
\begin{center}
\includegraphics[width = 8 cm]{Screenshot 2023-05-18 060516.png}
\end{center}
\end{frame}

\begin{frame}[t]{Data Association: The Assignment Problem}
\begin{itemize}
\item The Hungarian Algorithm is used to optimally solve the assignment problem represented by the cost matrix.
\begin{columns}[onlytextwidth]
\column{0.5\textwidth} 
\includegraphics[width = 5 cm]{Screenshot 2023-05-18 062758.png}
\column{0.5\textwidth} 
\includegraphics[width = 5 cm]{Screenshot 2023-05-18 062850.png}
\end{columns}


\begin{columns}[onlytextwidth]
\column{0.7\textwidth} 
The Hungarian Algorithm manages to reduce the computational complexity of the Assignment Problem from $O(n!)$ to $O(n^3)$.
\column{0.3\textwidth} 
\includegraphics[width = 3.5 cm]{1_yiyfZodqXNwMouC0-B0Wlg.png}
\end{columns}

\end{itemize}
\end{frame}

\begin{frame}{Track Maintenance}
\begin{itemize}
\item Objects can leave the video frame or become occluded for brief or long periods. we need to define the maximum number of frames without assigned detections, $T_{Lost}$, before deleting a track.
	
\item Additionally, SORT requires an object to be detected in two consecutive frames before confirming a track.
\end{itemize}
\end{frame}

\begin{frame}{Overview of the Pipeline}
\begin{center}
\includegraphics[width = 11 cm]{Overview-of-the-object-tracking-SORT-algorithm.png}
\end{center}
\end{frame}

\begin{frame}[t]{Performance Analysis: The CLEAR MOT Metrics}
The CLEAR multi-object tracking metrics provide a standard set of tracking metrics to evaluate
the quality of tracking algorithm
\begin{columns}[onlytextwidth]
\column{0.5\textwidth} 
	\begin{itemize}
	\item Multi-Object Tracking Accuracy (MOTA)
	\item Multi-Object Tracking Precision (MOTP)
	\item Mostly Tracked
	\item Partially Tracked 
	\item Mostly Lost     
	\item False Positive    
	\end{itemize}
\column{0.5\textwidth} 
	\begin{itemize}   
	\item False Negative    
	\item Recall    
	\item Precision    
	\item False Track Rate    
	\item ID Switches    
	\item Fragmentations
	\end{itemize}
\end{columns}
\end{frame}

\begin{frame}[t]{Performance Analysis: The CLEAR MOT Metrics}
Below is the evaluation of these metrics on our implementation:
\begin{center}
\includegraphics[width = 11 cm]{11.png}

\includegraphics[width = 8 cm]{22.png}

\includegraphics[width = 8 cm]{2.png}
\end{center}
\end{frame}

\begin{frame}[t]{Pros \& Cons}
\begin{itemize}
\item Pros:
	\begin{itemize}
	\item efficient \& extremely fast 
	\item low computational complexity due to simple linear motion model
	\item achieves state-of-the-art performance in MOTA
	\item  low number of lost targets in comparison to the other methods.
	\end{itemize}
\item Cons:
	\begin{itemize}
	\item high number of identity switches when subjected to long-term occlusions
	\item simplicity of motion model is problematic at low frame rates
	\end{itemize}
\end{itemize}
\end{frame}

\begin{frame}{Applications}
\begin{itemize}
\item Object tracking in video surveillance
\item Autonomous driving and vehicle tracking
\item Augmented reality and virtual reality applications
\item Sports analysis and player tracking
\item Robotics and unmanned aerial vehicles (UAVs) tracking
\end{itemize}
\end{frame}

\begin{frame}
\huge \textbf{Thank you for your time!} 
\end{frame}
\end{document}